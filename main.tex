\documentclass[12pt,a4paper,oneside]{book}
\usepackage[utf8]{inputenc}
\usepackage{amsmath}
\usepackage{amsfonts}
\usepackage{amssymb}
%
\usepackage{mathptmx}
\usepackage[T1]{fontenc}
\usepackage{indentfirst}
\usepackage{geometry}
 \geometry{
 left=4cm,
 top=4cm,
 right=3cm,
 bottom=3cm,
 }

% Mengkompres cite
\usepackage[numbers,sort&compress]{natbib}

% Daftar pustaka jadi section
\usepackage{etoolbox}
\patchcmd{\thebibliography}{\chapter*}{\section*}{}{}

%change Chapter to BAB
\usepackage{fancyhdr}
\renewcommand{\contentsname}{DAFTAR ISI}
\renewcommand{\bibname}{Daftar Pustaka}

\renewcommand\thesection{\Alph{section}.}
\renewcommand\thesubsection{\thesection\arabic{subsection}}

\linespread{2}

\usepackage{titlesec}
\titleformat{\chapter}[display]   
{\centering\normalfont\large\bfseries}{\chaptertitlename\ \thechapter}{0pt}{\large}   
\titlespacing*{\chapter}{0pt}{-50pt}{40pt}

\begin{document}
\frontmatter
\tableofcontents

%\chapter{Pendahuluan}
\newpage
\mainmatter
\section{Latar Belakang Masalah}

Selama bertahun-tahun, telah banyak penelitian tentang \textit{Multiple Traveling Salesman Problem} (MTSP), tetapi semuanya tentang jalur kendaraan di jalan. Akan tetapi jika transportasi yang digunakan adalah \textit{Unmanned Aerial Vehicle} (UAV) atau kendaraan tak berawak, ada masalah yang perlu dipertimbangkan masalah tersebut adalah persimpangan jalur travelling salesman \cite{inproceedings}. Persimpangan jalur travelling salesman tersebut harus dihindari, yang dapat menghindari tabrakan UAV. UAV dapat diterapkan pada bidang militer dan sipil, seperti pengintaian dan pengawasan dalam perang, pemadaman api, penyemprotan pestisida dan pengiriman barang.
Saat ini, penelitian tentang MTSP jarang menyinggung masalah menghindari persilangan jalur. Menerapkan algoritma ini untuk pengiriman UAV akan menyebabkan tabrakan. Selain itu, sebagian besar algoritma genetika untuk MTSP menetapkan beberapa titik virtual untuk menyalin MTSP ke TSP \cite{shengping2002hybrid}. Namun, jika skala kota besar, efisiensi algoritma akan menurun dengan cepat \cite{zhang2014parallel}.

Ada banyak upaya untuk menggunakan GA untuk pengklasteran, juga diamati bahwa metode ini mencari lebih cepat daripada beberapa algoritma evolusioner lain yang digunakan untuk pengklasteran \cite{krishna1999genetic}. Kemampuan pencarian dari algoritma genetika dimanfaatkan untuk mencari pusat cluster yang sesuai di ruang fitur sedemikian rupa sehingga metrik kesamaan dari cluster yang dihasilkan dioptimalkan \cite{maii2000genetic}. Ada juga upaya untuk menggunakan metode paralel untuk TSP untuk meningkatkan efisiensi \cite{li2016parallel}. Selama bertahun-tahun, berbagai algoritma cerdas telah diperkenalkan: \textit{Neural Networks} \cite{song2015asynchronous,zhang2015universality,pan2012spiking}, algoritma genetika, pengelompokan. Algoritma jaringan syaraf tiruan adalah sejenis algoritma pencocokan pola yang mensimulasikan jaringan syaraf biologis dan algoritma genetika mensimulasikan pemrosesan evolusi biologis \cite{liu2015implementation,zeng2014spiking,xu2016probe,zhang2014efficient}.

Menggabungkan semua perspektif ini, dalam skripsi ini, kami mengembangkan algoritma baru yang menggabungkan algoritma \textit{K-means} dan algoritma genetika. Maksud dari algoritma \textit{K-means} adalah melakukan preprocessing titik-titik pada MTSP. Kami membagi titik menjadi $m$ cluster sesuai dengan distribusinya, dan menemukan titik pusat sebagai titik awal dari algoritma genetika. Terakhir, GA digunakan untuk memproses setiap titik cluster secara paralel. Dalam hal ini, masalah skala besar dibagi menjadi beberapa masalah kecil oleh algoritme kami, dan algoritme genetika menunjukkan kinerja yang sangat tinggi dalam memecahkan TSP skala kecil (atau optimasi kombinatorial lainnya). Kami telah melakukan beberapa pengujian pada algoritme kami menggunakan banyak contoh, dan hasilnya menunjukkan bahwa sebagian besar masalah yang disebutkan di atas dapat diselesaikan.

\section{Identifikasi Masalah}

Dari latar belakang yang telah diuraikan,
identifikasi masalahnya adalah sebagai berikut.
\begin{enumerate}
	\item Penggunaan $k$-means dan algoritma genetika untuk mencari solusi MTSP.
	\item Pembagian klaster dan rute terdekat tiap klaster seluruh SMP yang ada di Kabupaten Probolinggo.
\end{enumerate}

\section{Rumusan Masalah}

Berdasarkan identifikasi masalah, maka rumusan masalah yang akan dikaji dalam penelitian ini sebagai berikut.
\begin{enumerate}
    \item Bagaimana cara mencari solusi \textit{multiple traveling salesman problem} dengan $k$-means dan algoritma genetika?
    \item Bagaimana perbandingan mencari solusi MTSP dengan menggunakan $k$-means dan Algoritma Genetika dibandingkan dengan menggunakan Algoritma Genetika tanpa menggunakan $k$-means?
\end{enumerate}

\section{Tujuan Penelitian}

Berdasarkan rumusan masalah di atas, tujuan dari skripsi ini yaitu untuk:
\begin{enumerate}
	\item Mengetahui cara kerja \textit{k-means clustering}.
	\item Mengetahui langkah-langkah algoritma genetika dalam mencari rute tercepat.
	\item Mengetahui cara menemukan solusi \textit{multiple traveling salesman problem} dengan \textit{k-means} dan algoritma genetika.
	\item Mengetahui perbandingan antara mencari solusi \textit{multiple traveling salesman problem} dengan \textit{k-means} dan algoritma genetika dibandingkan dengan tanpa menggunakan \textit{k-means}?
\end{enumerate}


% sudah direvisi ibu shofia

\section{Manfaat Penelitian}

Manfaat dari penelitian ini yaitu:
\begin{enumerate}
	\item Bagi Penulis, mengetahui cara menyelesaikan kasus \textit{Multiple Traveling Salesman Problem} yang telah dipelajari yaitu dengan menggunakan metode $K$-Means \textit{Clustering} dan Algoritma Genetika serta penulis dapat mengembangkan ilmu pemorgraman python pada komputer.

	\item Bagi Program Studi Pendidikan Matematika, menambah ilmu mengenai metode optimasi dan pencarian rute terdekat yang dapat diterapkan serta dipelajari kembali oleh mahasiswa pendidikan matematika untuk tahun-tahun selanjutnya, serta mengetahui rute-rute terdekat untuk menuju semua lokasi SMP di Kabupaten Probolinggo.
	
	\item Bagi Masyarakat, dapat menggunakan metode tersebut untuk menyelesaikan kasus \textit{Multiple Traveling Salesman Problem}, seperti penyebaran pestisida, pengintaian musuh pada militer, pendistribusian barang, dan lain-lain.
	
\end{enumerate}


\section{Definisi Konsep}

Dalam proposal ini, membahas tentang \textit{Multiple Traveling Salesman Problem} yang merupakan perluasan dari \textit{Travelling Salesman Problem} (TSP). TSP adalah permasalahan pencarian rute terpendek seorang salesman dari suatu kota ke kota lain tepat satu kali dan kembali ke kota yang sama. Sedangkan MTSP adalah permasalahan TSP oleh beberapa orang salesman dengan tujuan yang sudah dibagi.

Algoritma yang digunakan untuk membagi adalah $K$-means. $K$-means adalah jenis metode klasifikasi yang membagi item data menjadi beberapa klaster. Algoritma Genetika (AG) digunakan untuk pencarian rute terpendek. AG adalah algoritma yang digunakan untuk mencari solusi dari permasalahan dengan cara yang terispirasi dari teori evolusi. Dalam hal ini, algoritma genetika dapat juga digunakan untuk pencarian sebuah rute terpendek dalam sebuah kasus perjalanan. Untuk mengukur jarak antar 2 titik yang digunakan adalah metode euclidean distance.


\section{Penelitian Terdahulu}

Pada proposal ini, penulis mencantumkan 3 hasil penelitian yang memiliki relevansi atau keterkaitan dengan penelitian yang akan dilakukan sebagai berikut:

\begin{enumerate}
	\item Penelitian 1 - Raditya, P. M. R., dan Dewi, C. \cite{raditya2017optimasi}

Penelitian berjudul "Optimasi \textit{Multiple Travelling Salesman Problem} (M-TSP) Pada Penentuan Rute Optimal Penjemputan Penumpang \textit{Travel} Menggunakan Algoritme Genetika" penelitian ini membahas tentang permasalahan MTSP yaitu beberapa orang salesman yang akan berangkat dari kantor \textit{travel} menuju ke alamat penjemputan masing-masing penumpang. Pada permasalahan tersebut menggunakan representasi permutasi, proses reproduksi crossover dengan \textit{one cut point crossover}, proses mutasi dengan \textit{exchange mutation}, dan proses seleksi dengan \textit{elitism selection}.

	\item Penelitian 2 - Mayuliana, N. K., Kencana, E. N., dan Harini, L. P. I. \cite{mayuliana2015penyelesaian}

Penelitian berjudul "Penyelesaian Multitraveling Salesman Problem dengan Algoritma Genetika" penelitian ini mempelajari tentang kinerja algoritma genetika berdasarkan jarak minimum dan waktu pemrosesan yang diperlukan untuk 10 kali pengulangan untuk setiap kombinasi kota penjual.

	\item Penelitian 3 - Al-Khateeb, B., dan Yousif, M. \cite{al2019solving}

Penelitian berjudul "SOLVING MULTIPLE TRAVELING SALESMAN PROBLEM BY MEERKAT SWARM OPTIMIZATION ALGORITHM" dalam artikel ini mengusulkan algoritma metaheuristik yang disebut algoritma Meerkat Swarm Optimization (MSO) untuk memecahkan MTSP dan menjamin solusi berkualitas baik dalam waktu yang wajar untuk masalah kehidupan nyata.

\end{enumerate}





\section{Kajian Pustaka}

\textit{Multiple travelling salesman problem} (MTSP) adalah salah satu kombinatorial optimasi masalah, yang dapat didefinisikan sebagai berikut: Ada $m$ jumlah salesman yang harus melakukan perjalanan ke $n$ sejumlah kota dimulai dengan depot dan berakhir di depot yang sama, seperti kutipan dari artikel \cite{al2019comparative}. Selanjutnya para salesman harus melakukan perjalanan dari satu kota ke kota lain secara terus menerus tanpa mengulang kota mana saja yang telah dilintasi oleh para salesman dan mempertimbangkan jalur terpendek selama perjalanan tersebut.

$K$-Means adalah jenis metode klasifikasi tanpa pengawasan yang mempartisi item data menjadi satu atau lebih klaster, menurut agusta dalam artikelnya \cite{agusta2007k}. $K$-Means mencoba untuk memodelkan suatu dataset ke dalam klaster-klaster sehingga item-item data dalam suatu klaster memiliki karakteristik yang sama dan memiliki karakteristik yang berbeda dengan cluster lainnya.

Maulana menyebutkan dalam artikelnya algoritma adalah kumpulan perintah untuk menyelesaikan suatu masalah dan di selesaikan dengan cara sistematis, terstruktur dan logis. \cite{maulana2017pembelajaran}

Pada artikel Hermanto disebutkan bahwa algoritma genetika adalah algoritma yang digunakan untuk mencari solusi suatu permasalahan dengan cara yang lebih alami yang terispirasi dari teori evolusi  \cite{hermawanto2003algoritma}. Dalam hal ini, algoritma genetika dapat juga digunakan untuk pencarian sebuah rute terpendek dalam sebuah kasus perjalanan.

\section{Metode Penelitian}

Metode penelitian dalam proposal ini adalah metode penelitian dan pengembangan. Melalui metode ini diharapkan dapat mengembangkan algoritma yang diteliti.

Langkah-langkah yang akan dilakukan dalam penelitian:
\begin{enumerate}
	\item Menggunakan bahasa pemrograman \textit{Python} untuk mempermudah pengerjaan.
	
	\item Menyiapkan dataset random menggunakan "\textbackslash import random" pada \textit{Python} yang akan digunakan sebagai uji coba.
	
	\item Membagi titik pada MTSP menggunakan Algoritma $K$-Means menjadi $m$ klaster sesuai dengan distribusinya.
	
	\item Menggunakan Algoritma Genetika untuk menemukan rute terpendek untuk masing-masing klaster.
	
	\item Membandingkan penyelesaian MTSP dengan menggunakan algoritma genetika dan $k$-means dibandingkan dengan algoritma genetika tanpa $k$-means
	
	\item Menganalisi dan mengevaluasi data yang dihasilkan

\end{enumerate}

\section{Sistematika Penelitian}

Dengan menggunakan metode penelitian dan pengembangan maka penelitian akan mempu menghasilkan suatu produk yang memiliki nilai validasi tinggi, karena melalui serangkaian uji coba di lapangan dan divalidasi. Sistematika penelitian ini dibagi menjadi 4 tahap:

\begin{enumerate}

	\item \textbf{Perencanaan}
	Kegiatan yang dilakukan dalam tahap ini adalah sebagai berikut: penyusunan rancangan penelitian, penyiapan alat-alat penelitian, penetapan data penelitian, menyusun instrumen penelitian.
	
	\item \textbf{Pelaksanaan}
	Pada tahap ini peneliti selain pelaksana penelitian  juga mencari informasi data, yaitu membaca artikel penelitian sebelumnya yang berkaitan. Selain itu peneliti juga menyiapkan aplikasi yang akan digunakan untuk membantu perhitungan.
	
	\item \textbf{Analisis Data}
	Analisis data dilakukan setelah peneliti melakukan beberapa uji coba pada beberapa sampel data.
	
	\item \textbf{Evaluasi Semua Data}
	Data beberapa hasil rute yang tercepat, dan jarak minimal yang telah dianalisis kemudian dievaluasi sehingga diketahui bahwa metode ini merupakan metode yang efektif untuk menyelesaikan \textit{Multiple Traveling Salesman Problem} (MTSP).

\end{enumerate}

\section{\bibname}

\bibliographystyle{vancouver}
\bibliography{library}

\end{document}