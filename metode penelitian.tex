\section{Metode Penelitian}

Metode penelitian dalam proposal ini adalah metode penelitian dan pengembangan. Melalui metode ini diharapkan dapat mengembangkan algoritma yang diteliti.

Langkah-langkah yang akan dilakukan dalam penelitian:
\begin{enumerate}
	\item Menyiapkan dataset, dalam penelitian ini yang digunakan adalah dataset random
	\item Selanjutnya menentukan jumlah klaster yaitu sebanyak $n$ klaster
	
	Selanjutnya membagi data menggunakan algoritma \textit{k}-means
	
	\item Memilih sebanyak $n$ centroid secara acak
	\item Menghitung jarak data ke centroid dengan rumus \textit{euclidean distance}
	\begin{equation}
	d_{xy}=\sqrt{\sum_{i=1}^{n}(x_i-y_i)^{2}}
	\end{equation}
	\item Perbarui centroid dengan menghitung nilai rata-rata nilai pada masing-masing klaster
	\item Kembali ke tahapan 4 jika masih terdapat data yang berpindah klaster atau perubahan nilai centroid
	
	Selanjutnya melakukan proses TSP pada setiap klaster yang telah dibagi
	
	\item Membuat populasi awal secara random menggunakan data yang telah diklaster
	\item Melakukan reproduksi dengan crosover dengan peluang 0,95
	\item Melakukan mutasi pada data dengan peluang 0,01
	\item Seleksi dengan mode eliminasi
	
	\item Menentukan nilai fitness agar mendapatkan solusi akhir yang optimaldengan rumus:
	\begin{equation}
	fitness=\frac{10000}{RMSE}
	\end{equation}
	\item Iterasi dilakukan untuk generasi berikutnya sampai hasil yang dilakukan optimal atau mendekati optimal.
	
	\item Membandingkan penyelesaian MTSP dengan menggunakan algoritma genetika dan $k$-means dibandingkan dengan algoritma genetika tanpa $k$-means
	
	\item Menganalisi dan mengevaluasi data yang dihasilkan

\end{enumerate}