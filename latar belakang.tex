% Hasil revisi ibu shofia

\section{Latar Belakang Masalah}

Kabupaten Probobolinggo adalah salah satu dari beberapa kabupaten yang sedang berkembang di provinsi Jawa Timur. Banyak sekolah-sekolah menengah yang tersebar di Kabupaten Probolinggo. Oleh karena itu jika sebuah lembaga pendidikan yang sedang mengadakan kegiatan di Kabupaten Probolinggo dan ingin menyebarkan pamflet atau undangan diperlukanlah sebuah rute yang paling dekat agar dapat mempermudah perjalanan.

Selama bertahun-tahun, telah banyak penelitian tentang \textit{Multiple Traveling Salesman Problem} (MTSP). Berbagai metode telah digunakan untuk mencari solusi MTSP, salah satunya adalah Algoritma Genetika (AG), ada banyak upaya untuk menggunakan AG dalam pengklasteran, metode ini menemukan solusi lebih cepat daripada beberapa algoritma lain yang digunakan untuk pengklasteran \cite{krishna1999genetic}. Kemampuan menemukan solusi dari AG dimanfaatkan untuk mencari pusat klaster yang sesuai di ruang fitur sedemikian rupa sehingga kesamaan dari klaster yang dihasilkan dioptimalkan \cite{maii2000genetic}. Ada juga upaya untuk menggunakan metode paralel untuk TSP untuk meningkatkan efisiensi \cite{li2016parallel}. Namun, menurut Zhang efisiensi AG akan menurun dengan cepat jika digunakan pada skala kota besar \cite{zhang2014parallel}. 

Penggunaan AG dan dan algoritma \textit{k}-means adalah metode yang efektif untuk menyelesaikan MTSP, selain itu juga dapat menghindari persilangan antar salesman seperti yang dibahas oleh Lu pada artikelnya \cite{inproceedings}. Dari gabungan semua perspektif tersebut, dalam proposal ini, digunakanlah AG dan \textit{k}-means untuk menyelesaikan kasus pembagian klaster dan pencarian rute terdekat tiap klaster di seluruh SMP di Kabupaten Probolinggo.