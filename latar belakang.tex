\section{Latar Belakang Masalah}

Selama bertahun-tahun, telah banyak penelitian tentang \textit{Multiple Traveling Salesman Problem} (MTSP), tetapi semuanya tentang jalur kendaraan di jalan. Akan tetapi jika transportasi yang digunakan adalah \textit{Unmanned Aerial Vehicle} (UAV) atau kendaraan tak berawak, ada masalah yang perlu dipertimbangkan masalah tersebut adalah persimpangan jalur travelling salesman seperti yang dibahas pada artikel \cite{inproceedings}. Persimpangan jalur travelling salesman tersebut harus dihindari, agar dapat menghindari tabrakan UAV. UAV dapat diterapkan pada bidang militer dan sipil, seperti pengintaian dan pengawasan dalam perang, pemadaman api, penyemprotan pestisida dan pengiriman barang.
Saat ini, penelitian tentang MTSP jarang menyinggung masalah menghindari persilangan jalur. Penerapan Algoritma Genetika (AG) untuk pengiriman UAV akan menyebabkan tabrakan. Selain itu, sebagian besar AG untuk MTSP menetapkan beberapa titik untuk menyalin MTSP ke TSP seperti pada artikel \cite{shengping2002hybrid}. Namun, jika skala kota besar, efisiensi AG akan menurun dengan cepat, menurut Zhang pada artikelnya \cite{zhang2014parallel}.

Ada banyak upaya untuk menggunakan AG untuk pengklasteran, juga diamati bahwa metode ini menemukan solusi lebih cepat daripada beberapa algoritma lain yang digunakan untuk pengklasteran \cite{krishna1999genetic}. Kemampuan menemukan solusi dari AG dimanfaatkan untuk mencari pusat klaster yang sesuai di ruang fitur sedemikian rupa sehingga kesamaan dari klaster yang dihasilkan dioptimalkan \cite{maii2000genetic}. Ada juga upaya untuk menggunakan metode paralel untuk TSP untuk meningkatkan efisiensi \cite{li2016parallel}. Selama bertahun-tahun, berbagai algoritma cerdas telah diperkenalkan seperti  \textit{Neural Networks} \cite{song2015asynchronous,zhang2015universality,pan2012spiking}, algoritma genetika, pengelompokan. Algoritma jaringan syaraf tiruan adalah sejenis algoritma pencocokan pola yang mensimulasikan jaringan syaraf biologis dan algoritma genetika mensimulasikan pemrosesan evolusi biologis \cite{liu2015implementation,zeng2014spiking,xu2016probe,zhang2014efficient}.

Dari gabungan semua perspektif tersebut, dalam proposal ini, diusulkan sebuah algoritma yang menggabungkan algoritma $K$-means dan algoritma genetika. Algoritma $K$-means adalah melakukan preprocessing titik-titik pada MTSP. Titik-titik yang ada dibagi menjadi $m$ klaster sesuai dengan distribusinya, dan dicari titik pusat sebagai titik awal dari algoritma genetika. Terakhir, AG digunakan untuk menemukan rute terpendek untuk masing-masing klaster.% Dalam hal ini, masalah skala besar dibagi menjadi beberapa masalah kecil oleh algoritme kami, dan algoritme genetika menunjukkan kinerja yang sangat tinggi dalam memecahkan TSP skala kecil (atau optimasi kombinatorial lainnya). Kami telah melakukan beberapa pengujian pada algoritme kami menggunakan banyak contoh, dan hasilnya menunjukkan bahwa sebagian besar masalah yang disebutkan di atas dapat diselesaikan.