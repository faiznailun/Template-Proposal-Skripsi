\section{Latar Belakang Masalah}

Selama bertahun-tahun, telah banyak penelitian tentang \textit{Multiple Traveling Salesman Problem} (MTSP), tetapi semuanya tentang jalur kendaraan di jalan. Akan tetapi jika transportasi yang digunakan adalah \textit{Unmanned Aerial Vehicle} (UAV) atau kendaraan tak berawak, ada masalah yang perlu dipertimbangkan masalah tersebut adalah persimpangan jalur travelling salesman \cite{inproceedings}. Persimpangan jalur travelling salesman tersebut harus dihindari, yang dapat menghindari tabrakan UAV. UAV dapat diterapkan pada bidang militer dan sipil, seperti pengintaian dan pengawasan dalam perang, pemadaman api, penyemprotan pestisida dan pengiriman barang.
Saat ini, penelitian tentang MTSP jarang menyinggung masalah menghindari persilangan jalur. Menerapkan algoritma ini untuk pengiriman UAV akan menyebabkan tabrakan. Selain itu, sebagian besar algoritma genetika untuk MTSP menetapkan beberapa titik virtual untuk menyalin MTSP ke TSP \cite{shengping2002hybrid}. Namun, jika skala kota besar, efisiensi algoritma akan menurun dengan cepat \cite{zhang2014parallel}.

Ada banyak upaya untuk menggunakan GA untuk pengklasteran, juga diamati bahwa metode ini mencari lebih cepat daripada beberapa algoritma evolusioner lain yang digunakan untuk pengklasteran \cite{krishna1999genetic}. Kemampuan pencarian dari algoritma genetika dimanfaatkan untuk mencari pusat cluster yang sesuai di ruang fitur sedemikian rupa sehingga metrik kesamaan dari cluster yang dihasilkan dioptimalkan \cite{maii2000genetic}. Ada juga upaya untuk menggunakan metode paralel untuk TSP untuk meningkatkan efisiensi \cite{li2016parallel}. Selama bertahun-tahun, berbagai algoritma cerdas telah diperkenalkan: \textit{Neural Networks} \cite{song2015asynchronous,zhang2015universality,pan2012spiking}, algoritma genetika, pengelompokan. Algoritma jaringan syaraf tiruan adalah sejenis algoritma pencocokan pola yang mensimulasikan jaringan syaraf biologis dan algoritma genetika mensimulasikan pemrosesan evolusi biologis \cite{liu2015implementation,zeng2014spiking,xu2016probe,zhang2014efficient}.

Menggabungkan semua perspektif ini, dalam skripsi ini, kami mengembangkan algoritma baru yang menggabungkan algoritma \textit{K-means} dan algoritma genetika. Maksud dari algoritma \textit{K-means} adalah melakukan preprocessing titik-titik pada MTSP. Kami membagi titik menjadi $m$ cluster sesuai dengan distribusinya, dan menemukan titik pusat sebagai titik awal dari algoritma genetika. Terakhir, GA digunakan untuk memproses setiap titik cluster secara paralel. Dalam hal ini, masalah skala besar dibagi menjadi beberapa masalah kecil oleh algoritme kami, dan algoritme genetika menunjukkan kinerja yang sangat tinggi dalam memecahkan TSP skala kecil (atau optimasi kombinatorial lainnya). Kami telah melakukan beberapa pengujian pada algoritme kami menggunakan banyak contoh, dan hasilnya menunjukkan bahwa sebagian besar masalah yang disebutkan di atas dapat diselesaikan.