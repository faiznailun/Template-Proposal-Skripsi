\section{Definisi Konsep}

Dalam proposal ini, membahas tentang \textit{Multiple Traveling Salesman Problem} yang merupakan perluasan dari \textit{Travelling Salesman Problem} (TSP). TSP adalah permasalahan pencarian rute terpendek seorang \textit{salesman} dari suatu kota ke kota lain tepat satu kali dan kembali ke kota yang sama. Sedangkan MTSP adalah permasalahan TSP oleh beberapa orang \textit{salesman} dengan tujuan yang sudah dibagi.

Algoritma yang digunakan untuk membagi adalah $K$-means. $K$-means adalah jenis metode klasifikasi yang membagi item data menjadi beberapa klaster. Algoritma Genetika (AG) digunakan untuk pencarian rute terpendek. AG adalah algoritma yang digunakan untuk mencari solusi dari permasalahan dengan cara yang terispirasi dari teori evolusi. Dalam hal ini, algoritma genetika dapat juga digunakan untuk pencarian sebuah rute terpendek dalam sebuah kasus perjalanan. Untuk mengukur jarak antar 2 titik yang digunakan adalah metode \textit{euclidean distance}.
