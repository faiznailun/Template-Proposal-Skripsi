\section{Sistematika Penelitian}

Agar penulisan dalam penelitian yang diusulkan lebih terarah,
maka diperlukan sistematika penelitian.
Terkait hal tersebut,
sistematika penulisan dalam penelitian yang dilakukan nantinya adalah sebagai berikut.

\begin{enumerate}[label=]

	\item BAB I PENDAHULUAN 
	\begin{enumerate}[label=\Alph*.]
		\item Latar Belakang Masalah
		\item Rumusan Masalah
		\item Manfaat penelitian
		\item Tujuan Penelitian dan Pengembangan
		\item Batasan Masalah Penelitian
	\end{enumerate}

	\item BAB II KAJIAN PUSTAKA 
	\begin{enumerate}[label=\Alph*.]
		\item Penelitian relevan
		\item Dasar Teori
	\end{enumerate}

	\item BAB III KERANGKA TEORITIK DAN PENGEMBANGAN 
	\begin{enumerate}[label=\Alph*.]
		\item Model Penelitian dan Pengembangan
		\item Prosedur Penelitian dan Pengembangan
	\end{enumerate}

	\item BAB IV HASIL 
	\begin{enumerate}[label=\Alph*.]
		\item Penyajian Data Uji Coba
		\item Analisis Data
		\item Revisi Produk
	\end{enumerate}

	\item BAB V PENUTUP 
	\begin{enumerate}[label=\Alph*.]
		\item Kesimpulan
		\item Saran
	\end{enumerate}

\end{enumerate}


%penelitian akan mempu menghasilkan suatu produk yang memiliki nilai validasi tinggi, karena melalui serangkaian uji coba di lapangan dan divalidasi. Sistematika penelitian ini dibagi menjadi 4 tahap:
%
%\begin{enumerate}
%
%	\item Perencanaan
%	
%	Kegiatan yang dilakukan dalam tahap ini adalah sebagai berikut: penyusunan rancangan penelitian, penyiapan alat-alat penelitian, penetapan data penelitian, menyusun instrumen penelitian.
%	
%	\item Pelaksanaan
%	
%	Pada tahap ini peneliti selain pelaksana penelitian  juga mencari informasi data, yaitu membaca artikel penelitian sebelumnya yang berkaitan. Selain itu peneliti juga menyiapkan aplikasi yang akan digunakan untuk membantu perhitungan.
%	
%	\item Analisis Data
%	
%	Analisis data dilakukan setelah peneliti melakukan beberapa uji coba pada beberapa sampel data.
%	
%	\item Evaluasi Semua Data
%	
%	Data yang telah dianalisis kemudian dievaluasi sehingga diketahui bahwa metode ini merupakan metode yang efektif untuk menyelesaikan \textit{Multiple Traveling Salesman Problem} (MTSP).

%\end{enumerate}