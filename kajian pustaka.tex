\section{Kajian Pustaka}

\textit{Multiple travelling salesman problem} (MTSP) adalah salah satu kombinatorial optimasi masalah, yang dapat didefinisikan sebagai berikut: Ada $m$ jumlah salesman yang harus melakukan perjalanan ke $n$ sejumlah kota dimulai dengan depot dan berakhir di depot yang sama, seperti kutipan dari artikel \cite{al2019comparative}. Selanjutnya para salesman harus melakukan perjalanan dari satu kota ke kota lain secara terus menerus tanpa mengulang kota mana saja yang telah dilintasi oleh para salesman dan mempertimbangkan jalur terpendek selama perjalanan tersebut.

$K$-Means adalah jenis metode klasifikasi tanpa pengawasan yang mempartisi item data menjadi satu atau lebih klaster, menurut agusta dalam artikelnya \cite{agusta2007k}. $K$-Means mencoba untuk memodelkan suatu dataset ke dalam klaster-klaster sehingga item-item data dalam suatu klaster memiliki karakteristik yang sama dan memiliki karakteristik yang berbeda dengan cluster lainnya.

Maulana menyebutkan dalam artikelnya algoritma adalah kumpulan perintah untuk menyelesaikan suatu masalah dan di selesaikan dengan cara sistematis, terstruktur dan logis. \cite{maulana2017pembelajaran}

Pada artikel Hermanto disebutkan bahwa algoritma genetika adalah algoritma yang digunakan untuk mencari solusi suatu permasalahan dengan cara yang lebih alami yang terispirasi dari teori evolusi  \cite{hermawanto2003algoritma}. Dalam hal ini, algoritma genetika dapat juga digunakan untuk pencarian sebuah rute terpendek dalam sebuah kasus perjalanan.