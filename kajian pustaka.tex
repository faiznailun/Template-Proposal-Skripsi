% sudah direvisi ibu shofia

\section{Kajian Pustaka}

\subsection{\textit{Multiple Traveling Salesman Problem}}

Menurut Al-Omeer dan Ahmed, \textit{Multiple Travelling Salesman Problem} (MTSP) adalah salah satu kombinatorial optimasi masalah, yang dapat didefinisikan sebagai berikut: Ada $m$ jumlah salesman yang harus melakukan perjalanan ke $n$ sejumlah kota dimulai dengan depot dan berakhir di depot yang sama \cite{al2019comparative}. Selanjutnya para salesman harus melakukan perjalanan dari satu kota ke kota lain secara terus menerus tanpa mengulang kota mana saja yang telah dilintasi oleh para salesman dan mempertimbangkan jalur terpendek selama perjalanan tersebut.

Contoh solusi MTSP:

\begin{figure}[h!]
  \centering
  \includegraphics[width=0.5\textwidth]{Picture1.png}
  \caption{Solusi MTSP dengan membagi menjadi 6 klaster}
\end{figure}

\begin{figure}[h!]
  \centering
  \includegraphics[width=0.5\textwidth]{Picture2.png}
  \caption{Solusi MTSP dengan membagi menjadi 5 klaster}
\end{figure}

\subsection{Algoritma}

Maulana menyebutkan dalam artikelnya algoritma adalah kumpulan perintah untuk menyelesaikan suatu masalah dan diselesaikan dengan cara sistematis, terstruktur dan logis \cite{maulana2017pembelajaran}

\subsection{Algoritma $k$-means}

$K$-Means adalah jenis metode klasifikasi tanpa pengawasan yang mempartisi item data menjadi satu atau lebih klaster \cite{agusta2007k}. $K$-Means mencoba untuk memodelkan suatu dataset ke dalam klaster-klaster sehingga item-item data dalam suatu klaster memiliki karakteristik yang sama dan memiliki karakteristik yang berbeda dengan cluster lainnya.

Menurut S Monalisa \cite{monalisa2018klasterisasi} tahapan mengklaster menggunakan algoritma \textit{k}-means adalah sebagai berikut:

\begin{enumerate}
	\item Tentukan jumlah klaster
	\item Pilih centroid awal secara acak sesuai jumlah klaster
	\item Hitung jarak data ke centroid dengan rumus \textit{euclidean distance}
	\begin{equation}
	d_{xy}=\sqrt{\sum_{i=1}^{n}(x_i-y_i)^{2}}
	\end{equation}
	\item Perbarui centroid dengan menghitung nilai rata-rata nilai pada masing-masing klaster
	\item Kembali ke tahapan 3 jika masih terdapat data yang berpindah klaster atau perubahan nilai centroid
\end{enumerate}

\subsection{Algoritma Genetika}

Pada artikel Hermanto disebutkan bahwa algoritma genetika adalah algoritma yang digunakan untuk mencari solusi suatu permasalahan dengan cara yang lebih alami yang terispirasi dari teori evolusi  \cite{hermawanto2003algoritma}. Dalam hal ini, algoritma genetika dapat juga digunakan untuk pencarian sebuah rute terpendek dalam sebuah kasus perjalanan.

Menurut Armanda RS \cite{armanda2016penerapan} dalam artikelnya menyampaikan penyelesaian masalah menggunakan algoritma genetika memerlukan beberapa tahapan sebagai berikut:

\begin{enumerate}
	\item Membuat populasi awal secara random, dalam penelitian ini yang digunakan adalah data yang telah diklaster menggunakan algoritma \textit{k}-means
	\item Melakukan reproduksi dengan crosover dan mutasi pada pembentukan awal populasi
	\item Seleksi dengan metode elitism
	\item Menentukan nilai fitness agar mendapatkan solusi akhir yang optimal. Berikut merupakan persamaan perhitungan dalam mengetahui nilai fitness pada metode algoritma genetika
	\begin{equation}
	fitness=\frac{10000}{RMSE}
	\end{equation}
	\item Iterasi dilakukan untuk generasi berikutnya.
\end{enumerate}