\documentclass[12pt,a4paper,oneside]{book}
\usepackage[utf8]{inputenc}
\usepackage{graphicx} %input gambar
\usepackage{amsmath}
\usepackage{amsfonts}
\usepackage{amssymb}
%
\usepackage{mathptmx}
\usepackage[T1]{fontenc}

% Garis tepi (margin)
\usepackage{geometry}
 \geometry{
 left=4cm,
 top=4cm,
 right=3cm,
 bottom=3cm,
 }

% mengatur 1 cm baris pertama paragraf
\usepackage{indentfirst}
\setlength{\parindent}{1cm}

% Mengkompres cite
\usepackage[numbers]{natbib}

% Mengatur header dan foother
\usepackage{fancyhdr}
\pagestyle{fancy}
\fancyhead[L]{}  %mengosongkan head kiri
\fancyhead[R]{\thepage}  %memberi nomor page head kanan
\renewcommand{\headrulewidth}{0pt} %menghilangkan garis
\fancyfoot{} % menghilangkan footer

% change name Daftar Isi, Daftar Pustaka jadi section
\renewcommand{\contentsname}{DAFTAR ISI}
\renewcommand{\bibsection}{\section{Daftar Pustaka}}

% change languange dan pemenggalan kata
\usepackage[indonesian]{babel}
\sloppy

% numbering on section
\renewcommand\thesection{\Alph{section}.}
\renewcommand\thesubsection{\thesection\arabic{subsection}}

% baris antar paragraf
\linespread{1.5}

% mengatur title section
\usepackage{titlesec}
\titleformat{\chapter}[display]   
{\centering\normalfont\large\bfseries}{\chaptertitlename\ \thechapter}{0pt}{\large}   
\titlespacing*{\chapter}{0pt}{-50pt}{40pt}
% ukuran section dan subsection
\titleformat{\section}
  {\normalfont\fontsize{12}{15}\bfseries}{\thesection}{1em}{}
\titleformat{\subsection}
  {\normalfont\fontsize{12}{15}\bfseries}{\thesubsection}{1em}{}

% untuk modif numbering
\usepackage{enumitem}

% link
\usepackage{hyperref}