\documentclass[12pt,a4paper,oneside]{book}
\usepackage[utf8]{inputenc}
\usepackage{graphicx}
\usepackage{amsmath}
\usepackage{amsfonts}
\usepackage{amssymb}
%
\usepackage{mathptmx}
\usepackage[T1]{fontenc}
\usepackage{indentfirst}
\usepackage{geometry}
 \geometry{
 left=4cm,
 top=4cm,
 right=3cm,
 bottom=3cm,
 }

% Mengkompres cite
\usepackage[numbers,sort&compress]{natbib}

% Mengatur header dan foother
\usepackage{fancyhdr}
\pagestyle{fancy}
\fancyhead[L]{}  %mengosongkan head kiri
\fancyhead[R]{\thepage}  %memberi nomor page head kanan
\renewcommand{\headrulewidth}{0pt} %menghilangkan garis
\fancyfoot{} % menghilangkan footer

% change Chapter to BAB
\renewcommand{\contentsname}{DAFTAR ISI}
\renewcommand{\bibsection}{\section{Daftar Pustaka}}

% change languange dan pemenggalan kata
\usepackage[indonesian]{babel}

% numbering on section
\renewcommand\thesection{\Alph{section}}
\renewcommand\thesubsection{\thesection\arabic{subsection}}

% baris antar paragraf
\linespread{1.5}

% mengatur title section
\usepackage{titlesec}
\titleformat{\chapter}[display]   
{\centering\normalfont\large\bfseries}{\chaptertitlename\ \thechapter}{0pt}{\large}   
\titlespacing*{\chapter}{0pt}{-50pt}{40pt}
% memberi titik pada numbering section
\usepackage{secdot}
