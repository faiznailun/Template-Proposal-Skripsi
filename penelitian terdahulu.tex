\section{Penelitian Terdahulu}

Ada beberapa hasil penelitian sebelumnya yang memiliki keterkaitan dengan penelitian ini. Penelitian berjudul "Optimasi \textit{Multiple Travelling Salesman Problem} (M-TSP) Pada Penentuan Rute Optimal Penjemputan Penumpang \textit{Travel} Menggunakan Algoritme Genetika" karangan Raditya, P. M. R., dan Dewi, C. \cite{raditya2017optimasi}. Penelitian tersebut membahas tentang permasalahan MTSP yaitu beberapa orang salesman yang akan berangkat dari kantor \textit{travel} menuju ke alamat penjemputan masing-masing penumpang. Pada permasalahan tersebut menggunakan representasi permutasi, proses reproduksi crossover dengan \textit{one cut point crossover}, proses mutasi dengan \textit{exchange mutation}, dan proses seleksi dengan \textit{elitism selection}.

Mayuliana, N. K., Kencana, E. N., dan Harini, L. P. I. Dalam artikelnya yang berjudul “Penyelesaian Multitraveling Salesman Problem dengan Algoritma Genetika” \cite{mayuliana2015penyelesaian}, mempelajari tentang kinerja algoritma genetika berdasarkan jarak minimum dan waktu pemrosesan yang diperlukan untuk 10 kali pengulangan untuk setiap kombinasi kota penjual.

Artikel karangan Al-Khateeb, B., dan Yousif, M. berjudul "SOLVING MULTIPLE TRAVELING SALESMAN PROBLEM BY MEERKAT SWARM OPTIMIZATION ALGORITHM" \cite{al2019solving} dalam artikel ini mengusulkan algoritma metaheuristik yang disebut algoritma Meerkat Swarm Optimization (MSO) untuk memecahkan MTSP dan menjamin solusi berkualitas baik dalam waktu yang wajar untuk masalah kehidupan nyata.
