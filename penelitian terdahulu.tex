\section{Penelitian Terdahulu}

Pada proposal ini, penulis mencantumkan 3 hasil penelitian yang memiliki relevansi atau keterkaitan dengan penelitian yang akan dilakukan sebagai berikut:

\begin{enumerate}
	\item Penelitian 1 - Raditya, P. M. R., dan Dewi, C. \cite{raditya2017optimasi}

Penelitian berjudul "Optimasi \textit{Multiple Travelling Salesman Problem} (M-TSP) Pada Penentuan Rute Optimal Penjemputan Penumpang \textit{Travel} Menggunakan Algoritme Genetika" penelitian ini membahas tentang permasalahan MTSP yaitu beberapa orang salesman yang akan berangkat dari kantor \textit{travel} menuju ke alamat penjemputan masing-masing penumpang. Pada permasalahan tersebut menggunakan representasi permutasi, proses reproduksi crossover dengan \textit{one cut point crossover}, proses mutasi dengan \textit{exchange mutation}, dan proses seleksi dengan \textit{elitism selection}.

	\item Penelitian 2 - Mayuliana, N. K., Kencana, E. N., dan Harini, L. P. I. \cite{mayuliana2015penyelesaian}

Penelitian berjudul "Penyelesaian Multitraveling Salesman Problem dengan Algoritma Genetika" penelitian ini mempelajari tentang kinerja algoritma genetika berdasarkan jarak minimum dan waktu pemrosesan yang diperlukan untuk 10 kali pengulangan untuk setiap kombinasi kota penjual.

	\item Penelitian 3 - Al-Khateeb, B., dan Yousif, M. \cite{al2019solving}

Penelitian berjudul "SOLVING MULTIPLE TRAVELING SALESMAN PROBLEM BY MEERKAT SWARM OPTIMIZATION ALGORITHM" dalam artikel ini mengusulkan algoritma metaheuristik yang disebut algoritma Meerkat Swarm Optimization (MSO) untuk memecahkan MTSP dan menjamin solusi berkualitas baik dalam waktu yang wajar untuk masalah kehidupan nyata.

\end{enumerate}



